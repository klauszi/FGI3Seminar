\documentclass[presentation,xcolor=svgnames,9pt]{beamer}

\usepackage{ngerman}
\usepackage{amsthm}
\usepackage{lipsum}
\usepackage{capt-of}
\usepackage[noend]{algpseudocode}
\usepackage{amssymb}
\usepackage{amsmath}
\usepackage[utf8]{inputenc}
\usepackage{rotating}
\usepackage[dvips]{epsfig}
\usepackage{graphicx}
\usepackage[outdir=./]{epstopdf}
\usepackage{acronym}
\usepackage{tikz}
\usetikzlibrary{automata,positioning}
\usepackage{float}
\usepackage[T1]{fontenc}
\usepackage[ngerman]{babel}
\usepackage{color}
\usepackage{graphicx}
\usepackage{listings}
\usepackage{tikz}
\usetikzlibrary{fit,backgrounds}
\usetikzlibrary{calc}
\usepackage{algorithm}
\usepackage{multirow}
\usepackage{algorithmicx}
\usepackage{tikz}
% \usepackage{enumitem}
\usepackage{selinput}
\usepackage{ulem}

\setbeamercovered{transparent}
\tikzset{cross/.style={cross out, draw=black, minimum
size=2*(#1-\pgflinewidth), inner sep=0pt, outer sep=0pt},
%default radius will be 1pt. 
cross/.default={1pt}}

\usetikzlibrary{arrows,shapes,snakes,automata,backgrounds,petri}
%Leftrightarrow \setbeamercovered{again covered={\opaqueness<1->{15}}}
\lstset{basicstyle=\ttfamily,
  showstringspaces=false,
  commentstyle=\color{red},
  keywordstyle=\color{blue}
}

\newcommand{\mytab}[1]{%
	\begin{tabular}{@{}c@{}}
		#1
	\end{tabular}
}

\pgfdeclarelayer{bg}    % declare background layer
\pgfsetlayers{bg,main}  % set the order of the layers (main is the standard layer)

\mode<presentation>
{
	\usetheme{Malmoe}

	\usecolortheme{crane}
	\useinnertheme{circles}

	% Disable navigation bar
	\setbeamertemplate{navigation symbols}{}


}


\selectlanguage{ngerman}

\title[FGI3]{Verbesserte Schranke der Cover Time für COBRA-Prozesse auf
Graphen}
\author[Klaus-J. Ziegert]{Klaus-J. Ziegert}

\institute[]{Universität Hamburg\\
	Fakultät für Mathematik, Informatik und Naturwissenschaften\\
	Fachbereich Informatik}

\date{\today}

\logo{\includegraphics[height=0.5cm]{uhh_logo.png}}
\useoutertheme{infolines}

\begin{document}

	\begin{frame}
		\titlepage
	\end{frame}

	\section{Einleitung}

	\begin{frame}
	\frametitle{Motiviation: Random Walk Verbesserung}
	\end{frame}

	\begin{frame}
		\frametitle{COBRA Prozess}
		\begin{itemize}
			\item Man beachte einen ungerichteten und
				zusammenhängenden Graphen
				$G = (V,E)$ mit $n$ Knoten und
				$m$ Kanten
			\item Es sei $C$ eine Teilmenge der
				Knoten, die am Anfang
				informiert sind und die Nachricht
				versenden können
			\item In jeder Runde $i$ wählen die
				Knoten in $C_{i-1}$ zufällig und
				gleichverteilt $b \geq 1$
				Nachbarn (mit Wiederholung) aus
			\item Ausgewählte Knoten
				erhalten eine Nachricht in der nächsten Runde
				$C_{i+1}$; nicht
				ausgewählte Knoten verlieren u.U.
				die Information
			\item Sei $Hit_{C}(u)$ die Runde bei dem
				der
				Knoten $u$ das erste Mal
				benachrichtigt wurde, so ist die
				Cover Time: 
				\[
				Cover(C) =
				max\left\{
					Hit_{C}(v) : v \in V)
				\right\}
				\]
		\end{itemize}
	\end{frame}

	\begin{frame}
  \only<1-7>{
		\frametitle{Beispiel COBRA Prozess 
	}
		\begin{columns}
		\begin{column}{0.55\linewidth}
\begin{figure}
\begin{tikzpicture}[node distance=2cm,>=stealth',bend angle=45,auto]

  \tikzstyle{place}=[circle,thick,draw=black!75,minimum size=6mm]
  \tikzstyle{red place}=[place,draw=black!75,fill=red!75]
  \tikzstyle{every label}=[black]

  \begin{scope}

	\only<1>{\node [red place,tokens=1,label=left:$0$] (0) []
		{};}
	\only<2-3>{\node [red place,tokens=3,label=left:$0$] (0) []
		{};}
	\only<4>{\node [red place,tokens=1,label=left:$0$] (0) []
		{};}
	\only<5-7>{\node [red place,tokens=1,label=left:$0$] (0) []
		{};}

	\only<1>{\node [red place,tokens=1,label=below:$2$] (2)
		[right of=0] {};}
	\only<2-3>{\node [red place,tokens=3,label=below:$2$] (2)
		[right of=0] {};}
	\only<4>{\node [red place,tokens=0,label=below:$2$] (2)
		[right of=0] {};}
	\only<5>{\node [red place,tokens=0,label=below:$2$] (2)
		[right of=0] {};}
	\only<6>{\node [red place,tokens=1,label=below:$2$] (2)
		[right of=0] {};}
	\only<7>{\node [red place,tokens=0,label=below:$2$] (2)
		[right of=0] {};}

	\only<1-3>{\node [place,tokens=0,label=above:$1$] (1)
		[above of=2] {}};
	\only<4>{\node [place,tokens=4,label=above:$1$] (1)
		[above of=2] {}};
	\only<5>{\node [red place,tokens=1,label=above:$1$] (1)
		[above of=2] {}};
	\only<6-7>{\node [red place,tokens=1,label=above:$1$] (1)
		[above of=2] {}};

	\only<1-3>{\node [place,tokens=0,label=below:$3$] (3)
		[below of=2] {};}
	\only<4>{\node [place,tokens=1,label=below:$3$] (3)
		[below of=2] {};}
	\only<5>{\node [red place,tokens=1,label=below:$3$] (3)
		[below of=2] {};}
	\only<6-7>{\node [red place,tokens=0,label=below:$3$] (3)
		[below of=2] {}};

	\only<1-5>{\node [place,tokens=0,label=right:$4$] (4)
		[right of=1] {};}
	\only<6-7>{\node [red place,tokens=1,label=right:$4$] (4)
		[right of=1] {}};

	\only<1-6>{\node [place,tokens=0,label=right:$5$] (5)
		[right of=2] {};}
	\only<7>{\node [red place,tokens=1,label=right:$5$] (5)
		[right of=2] {};}
	\path (0) edge (1)
	(0) edge (2)
	(0) edge (3)
	(1) edge (4)
	(1) edge (2)
	(2) edge (5);
	\only<3>{
		\begin{pgfonlayer}{bg}
		\path (0) edge[->,red!50,line width=2.0pt] node
		{$2$} (1);
		\end{pgfonlayer}
	}
	\only<3>{
		\begin{pgfonlayer}{bg}
		\path (0) edge[->,red!50,line width=2.0pt]
		(3){};
		\end{pgfonlayer}
	}
		\only<3>{
		\begin{pgfonlayer}{bg}
			\path (2) edge[->,red!50,line width=2.0pt] (0) {};
		\end{pgfonlayer}
		}
	\only<3>{
		\begin{pgfonlayer}{bg}
			\path (2) edge[->,red!50,line width=2.0pt] node
			{$2$}(1) {}; 
		\end{pgfonlayer}
	}
  \end{scope}
\end{tikzpicture}
\caption{$C = \left\{ 0,2 \right\}$, $b = 3$}
\end{figure}
  \end{column}
	  \begin{column}{0.45\linewidth}
		  \only<1-5>{
		\begin{block}{
		\only<1>{(Runde 0)}\only<2>{(Runde 1 Phase
		1)}\only<3>{(Runde 1 Phase 2a)}\only<4>{(Runde 1
		Phase 2b)}\only<5>{(Runde 1 Phase 3)}
	}
		\only<1>{Knoten in $C_{0}$ hat eine Marke}
		\only<2>{Jede Marke teilt sich in $b$ Submarken}
		\only<3>{Jede Submarke wählt zufällig und
		gleichverteilt einen Nachbarn}
		\only<4>{Submarken gehen zum ausgewählten Nachbarn
		über}
		\only<5>{Submarken werden oder verschmelzen ggf. zu einer
		Marke}
	\end{block}
}
		  \begin{block}{History}
			  \begin{itemize}
				\item $C_{0} = \left\{ 0,2 \right\}$
				\only<5-7>{
				\item $C_{1} = \left\{
						0,1,2,3
					\right\}$
				}
				\only<6-7>{
				\item $C_{2} = \left\{
						0,1,2,4
					\right\}$
				}
				\only<7>{
				\item $C_{3} = \left\{
						0,1,4,5
					\right\}$
				\item $\dots$ 
				\item $Cover(C) = Hit_{C}(5) = 3$
				}

			  \end{itemize}
		  \end{block}
	  \end{column}

  \end{columns}
  }
	\end{frame}


	\begin{frame}
		\only<1-2>
		{
		\only<1>{\frametitle{Bekannte Schranken der
		$E[Cover(G)]$ für beliebige Graphen}}
		\only<2>{\frametitle{Neue obere Schranken der
		$E[Cover(G)]$ für beliebige Graphen}}
		\begin{figure}
		\begin{tikzpicture}
			\draw[->] (0,0) -- (11,0);
			\foreach \x in {0.8}
			\draw(\x cm,7pt) -- (\x cm, -3pt);
			\draw (0.8,0) node[below=3pt] {$0$};
			\draw (5,0)
			node[rectangle,above=1pt,draw=black,fill=red!
				75,
			minimum width=5cm]
			{$E[Cover(G)]$};
			\draw[->, line width=1pt] (2,1.5)
			node[above=3pt]
			{$\Omega\left( max\left(
				Diam(G),\log n \right) \right)$}-- (2,0) ;
			\only<1>{
			\draw (8,3) node[above=3pt]
			{\color{white}{\textbf{Theorem
			1.1:}}}; 
			\draw[->, line width=1pt] (10.5,1.5)
			node[above=3pt]
			{$\mathcal{O}\left(n^{11/4}\log n \right)$}--
			(10.5,0) ;
		}
			\only<2>{
			\draw[->, line width=1pt] (10.5,1.5)
			node[above=3pt]
			{\sout{$\mathcal{O}\left(n^{11/4}\log n
		\right)$}}--
			(10.5,0) ;
			\draw (8,3) node[above=3pt] {\textbf{Theorem
			1.1:}}; 
			\draw[->, line width=1pt] (8,2)
			node[above=3pt]
			{$\mathcal{O}\left(m + \left(
			d_{max}
			\right)^2\log n
		\right)$}--
			(8,0) ;
		}
		\end{tikzpicture}
		\caption{Schranken des Erwartungswertes der Cover
		Time für beliebige Graphen $G$}
	\end{figure}
}
	\end{frame}

	\begin{frame}
		\only<1-2>
		{
		\only<1>{\frametitle{Bekannte Schranken der 
			$E[Cover(G_{r})]$ für reguläre Graphen}}
		\only<2>{\frametitle{Neue obere Schranke für
			$E[Cover(G_{r})]$ für reguläre Graphen}}
		\begin{figure}
		\begin{tikzpicture}
			\draw[->] (0,0) -- (11,0);
			\foreach \x in {0.8}
			\draw(\x cm,7pt) -- (\x cm, -3pt);
			\draw (0.8,0) node[below=3pt] {$0$};
			\draw (5,0)
			node[rectangle,above=1pt,draw=black,fill=red!
				75,
			minimum width=5cm]
			{$E[Cover(G)]$};
			\draw[->, line width=1pt] (2,1.5)
			node[above=3pt]
			{$\Omega\left( max\left(
				Diam(G),\log n \right) \right)$}-- (2,0) ;
			\only<1>{
			\draw (8,3) node[above=3pt]
			{\color{white}{\textbf{Theorem
			1.2:}}}; 
			\draw[->, line width=1pt] (8.5,2.1)
			node[above=3pt]
			{
				$
			\begin{aligned}
				&\mathcal{O}\left( \frac{r^4}{
				\phi^2
				}\log^2 n
				\right) 
				&
				\\
				&\mathcal{O}\left( \frac{1}{\left(
				1-\lambda
				\right)^3}\log n
				\right) 
				&\text{, falls
				} 1 - \lambda > C
				\sqrt{\frac{\log n}{n}}\\
				& \color{white}{
				\mathcal{O}\left( \left( \frac{r}{\left(
				1-\lambda
			\right)} + r^2 \right) \log n 
		\right)}
				&
				\color{white}{
				\text{, falls
				} 1 - \lambda > C
				\sqrt{\frac{\log n}{n}}
			}\\
			\end{aligned}
				$
			} --
			(8.5,0) ;
		}
			\only<2>{
			\draw[->, line width=1pt] (8.5,2.1)
			node[above=3pt]
			{
				$
			\begin{aligned}
				&\mathcal{O}\left( \frac{r^4}{
				\phi^2
				}\log^2 n
				\right) 
				&
				\\
				&\mathcal{O}\left( \frac{1}{\left(
				1-\lambda
				\right)^3}\log n
				\right) 
				&
				\text{, falls
				} 1 - \lambda > C
				\sqrt{\frac{\log n}{n}}\\
				&\mathcal{O}\left( \left( \frac{r}{\left(
				1-\lambda
			\right)} + r^2 \right) \log n 
				\right) 
				&
				\text{, falls
				} 1 - \lambda > C
				\sqrt{\frac{\log n}{n}}\\
			\end{aligned}
				$
			} --
			(8.5,0) ;
			\draw (3.7,2.4) node[above=3pt] {\textbf{Theorem
			1.2:}}; 
		}
		\end{tikzpicture}
		\caption{Schranken des Erwartungswertes der Cover
			Time für
		$r$-reguläre Graphen $G$}
	\end{figure}
}
	\end{frame}

	\begin{frame}
		\frametitle{Resultate des Papers}
		\begin{block}{Theorem 1.1}
			Sei $G$ ein zusammenhängender Graph mit
			$n$ Knoten, $m$ Kanten und den
			Maximalgrad $d_{max}$. Die Cover Time
			$cover(u)$ eines COBRA-Prozesses mit $b =
			2$ auf dem Graphen $G$ ist m.h.W. für
			alle Knoten $u \in V$:
			\[\mathcal{O}\left( m + \left( d_{max}
			\right)^2 \log n \right)\]
		\end{block}
		\begin{block}{Theorem 1.2}
			Sei $G$ ein zusammenhängender $r$-regulärer Graph mit
			$n$ Knoten, der die Bedingung
			$1 - \lambda > C
			\sqrt{\frac{\log n}{n}}$ für eine
			angemessen große Konstante $C$ erfüllt. 
			 Die Cover Time
			$cover(u)$ eines COBRA-Prozesses mit $b =
			2$ auf dem Graphen $G$ ist m.h.W. für
			alle Knoten $u \in V$:
			\[\mathcal{O}\left( \left(
				\frac{r}{1-\lambda} + r^2
			\right) \log n \right)\]
		\end{block}
	\end{frame}

	\begin{frame}
		\frametitle{Vorgehen des Papers}
	\end{frame}

	\section{Dualer BIPS Prozess}

	\begin{frame}
		\frametitle{Vorgehen des Papers}
	\end{frame}

	\section{natürlicher Wachstum des BIPS Prozesses}

	\begin{frame}
	\end{frame}

	\section{COBRA Prozess mit Branching-Faktor kleiner als 2}

	\begin{frame}

	\end{frame}

	\section{Zusammenfassung}

\end{document}

